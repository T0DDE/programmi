\documentclass{article}
\usepackage[utf8]{inputenc}
\usepackage[T1]{fontenc}
\usepackage{amsmath}
\usepackage{amssymb}
\usepackage{graphicx}
\usepackage{hyperref}

\begin{document}

\section{Panorama mondiale energetico}

La \textbf{carbon intensity} è definita come il rapporto tra le tonnellate di CO\textsubscript{2} equivalente emesse e i miliardi di Prodotto Interno Lordo (PIL). Il PIL è in generale utilizzato come indicatore della ricchezza prodotta da un Paese.

Per mantenere il riscaldamento globale entro i limiti stabiliti dagli accordi internazionali sul clima (come l’Accordo di Parigi), occorre una riduzione della carbon intensity di circa \textbf{20,4\%}. Tale obiettivo richiede iniziative coordinate a livello globale: prima nasce un accordo politico, e solo successivamente si procede alla fase di \emph{execution}, cioè all’attuazione concreta.

Per raggiungere gli obiettivi del \textbf{Green Deal Europeo}, l’Unione Europea ha definito pacchetti di iniziative legislative quali:
\begin{itemize}
    \item \textbf{Fit for 55}: insieme di misure per ridurre del 55\% le emissioni entro il 2030;
    \item \textbf{REPowerEU}: piano successivo, volto ad accelerare la transizione energetica e ridurre la dipendenza dai combustibili fossili esteri.
\end{itemize}

Nel 2024 nasce il \textbf{PNIEC} (Piano Nazionale Integrato Energia e Clima), che fissa gli obiettivi italiani al 2030 riguardo:
\begin{enumerate}
    \item riduzione delle emissioni di CO\textsubscript{2} equivalente rispetto ai livelli del 1990;
    \item quota di \textbf{FER} (Fonti Energetiche Rinnovabili) nei consumi finali ed elettrici;
    \item sviluppo della capacità solare installata.
\end{enumerate}

Gli \textbf{scenari energetici} non sono previsioni: fissato l’obiettivo, gli scenari definiscono la traiettoria necessaria per raggiungerlo. In Italia, questi scenari sono elaborati da \textbf{Terna} (Transmission System Operator elettrico) insieme a \textbf{SNAM} (TSO del gas). Poiché le due società operano su vettori energetici differenti — elettrico e gas — la redazione congiunta del documento risulta complessa. Per normativa, i due TSO devono collaborare alla pianificazione.

\subsection{Sull'importazione di Gas}

\subsubsection{Il contesto ucraino}

L’Ucraina è attraversata da importanti gasdotti che permettono ai produttori, in particolare la Federazione Russa, di inviare gas verso l’Europa occidentale. L’Italia ha una produzione interna di idrocarburi molto bassa (principalmente tramite ENI, Ente Nazionale Idrocarburi), e quindi dipende fortemente dalle importazioni.

Negli ultimi anni:
\begin{itemize}
    \item le importazioni dalla Russia sono diminuite;
    \item sono aumentate le importazioni di \textbf{GNL} (Gas Naturale Liquefatto) proveniente dagli Stati Uniti;
    \item l’Italia rimane un paese poco diversificato nella propria strategia di approvvigionamento.
\end{itemize}

Nel 2021 il fabbisogno era pari a circa \textbf{76 miliardi di m\textsuperscript{3}} di gas, quasi interamente importati; nel 2024 il consumo si è ridotto a circa \textbf{62 miliardi di m\textsuperscript{3}}.

A differenza del gas, l’olio (petrolio) è più semplice e sicuro da trasportare su lunghe distanze.

\subsection{Il rialzo dei prezzi dell'energia}

Negli ultimi anni si è verificato un significativo aumento dei prezzi dell’energia. Poiché i mercati del gas e dell’elettricità sono fortemente \textit{correlati}, l’aumento del prezzo del gas si riflette quasi immediatamente su quello dell’energia elettrica.

\paragraph{Dinamiche recenti}

\begin{itemize}
    \item Nel 2021 il prezzo del gas inizia ad aumentare: in Italia si passa da circa 20 €/MWh a valori intorno ai 30 €/MWh.
    \item Nel 2020, durante la pandemia COVID-19, la domanda energetica era crollata e i prezzi erano molto bassi.
    \item L’aumento dei consumi energetici in Cina — paese con una domanda elevatissima — ha portato a un maggior utilizzo del metano come alternativa meno inquinante al carbone.
    \item Nel 2022 lo scoppio della guerra in Ucraina ha causato un ulteriore incremento dei prezzi.
\end{itemize}

\subsubsection{Aspetti ambientali e mercato della CO\textsubscript{2}}

Il sistema europeo di scambio delle emissioni, \textbf{ETS} (Emission Trading Scheme), stabilisce un prezzo per la CO\textsubscript{2} emessa. Parallelamente, il \textbf{PUN} (Prezzo Unico Nazionale) del mercato elettrico italiano, nato nel 2004 con la prima borsa elettrica, è oggi superato nei fatti dai prezzi zonali, che meglio riflettono le specificità territoriali.

Il PUN era stato introdotto per evitare differenze significative tra Nord e Sud, dove erano allora presenti principalmente impianti termoelettrici. Oggi è un indice di riferimento storico, non più rappresentativo delle dinamiche reali di mercato.

\paragraph{Caro energia}

Con “caro energia” si indicano i vari decreti del governo volti a contenere l’aumento dei costi energetici. Per favorire la diffusione di nuove tecnologie (es. fotovoltaico), lo Stato ricorre spesso agli incentivi, che vengono poi coperti attraverso la componente \textbf{ASOS} della bolletta elettrica, insieme agli \textbf{oneri di sistema} (ad esempio lo smantellamento delle centrali nucleari).

Poiché circa il 50\% della bolletta riguarda la materia prima, il governo è intervenuto sugli oneri di sistema, anche attraverso la legge di bilancio, che nel 2024 ha destinato circa \textbf{18 miliardi di euro} al caro energia. Ciò evidenzia la centralità del contenimento dei prezzi per la competitività del Paese.

\subsection{Come funziona la borsa elettrica}

L’\textbf{ARERA} (Autorità di Regolazione per Energia Reti e Ambiente), nata come AEEG nel 1995, disciplina il mercato energetico. Il decreto legislativo \textbf{79/1999} (cosiddetto ``Decreto Bersani'') ha avviato la liberalizzazione del mercato elettrico, ponendo fine al modello verticalmente integrato di ENEL.

Successivamente, il progetto \textbf{TIDE} (Testo Integrato Dispacciamento Elettrico) è stato avviato nel 2019 e la relativa delibera è stata approvata nel 2023.

\subsubsection{Definizione dei programmi di produzione}

All’interno del Gruppo GSE (Gestore dei Servizi Energetici), alcune società stipulano:
\begin{itemize}
    \item \textbf{contratti bilaterali}, che non passano direttamente attraverso il mercato;
    \item contratti sul mercato elettrico gestito dal \textbf{GME} (Gestore dei Mercati Energetici), che funge da controparte centrale.
\end{itemize}

Nei contratti bilaterali il GME non interviene nella definizione del prezzo, che può essere comunque indicizzato ai prezzi di mercato.

\subsubsection{Dispacciamento}

Il dispacciamento si divide in:
\begin{enumerate}
    \item \textbf{di merito}: basato sul costo marginale delle tecnologie;
    \item \textbf{passante}: legato a esigenze di sicurezza del sistema elettrico.
\end{enumerate}

\section{II Parte: Evoluzione del sistema e del mercato elettrico}

\textbf{Docente: Salvatore De Carlo}, responsabile di System Strategy and Positioning, che svolge attività assimilabile a una consulenza strategica interna.

Attualmente, la tecnologia marginale è ancora rappresentata dal gas naturale. Di conseguenza, il prezzo dell’elettricità è determinato dal costo del gas e dalle quote della CO\textsubscript{2}. Una formula approssimata del prezzo elettrico è:
\[
\text{Prezzo elettrico} \approx 2 \times \text{Prezzo gas} + \frac{\text{Prezzo CO}_2}{2}.
\]

L’Europa è un grande importatore di gas, sia tramite metanodotti sia tramite GNL (navi metaniere).

Tra il 2019 e il 2022, a causa della guerra, i costi di importazione del gas sono aumentati di circa \textbf{75 miliardi di euro}, una cifra comparabile agli investimenti necessari per realizzare il PNIEC entro il 2030. Se la transizione energetica fosse stata avviata prima, la dipendenza dal gas — e dunque questa spesa — si sarebbe notevolmente ridotta.

Il meccanismo del prezzo marginale funziona finché esiste una tecnologia più costosa che determina il prezzo. In un sistema composto solo da eolico e fotovoltaico, il prezzo marginale tenderebbe a zero: ciò non incentiverebbe nuovi investimenti. È quindi prevedibile un cambiamento nel modello di mercato, orientato verso:
\begin{itemize}
    \item \textbf{contratti di lungo termine};
    \item strumenti di stabilizzazione del prezzo e riduzione del rischio.
\end{itemize}

\subsubsection{Cambiamento climatico}

Il cambiamento climatico incide anche sul sistema energetico: scarsità d’acqua per il raffreddamento e l’idroelettrico, eventi meteorologici estremi, alluvioni e ondate di calore possono compromettere la continuità del servizio elettrico e l’integrità delle infrastrutture.










\end{document}
