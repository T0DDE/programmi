
%   Il progetto nasce dal template per il frontespizio realizzato da Marco Antonio Corallo, che ringrazio. Seguono alcuni commenti per evidenziare la presenza di alcuni pacchetti che mi sono stati utili per la stesura della tesi. Chiaramente, dipende tutto dal tipo di lavoro che uno vuole eseguire, che determina anche le diverse esigenze. Durante la stesura ho passato molto tempo su siti e forum a cercare di risolvere alcuni probelmi di formattazione, ma in generale Latex è stato piuttosto versatile. 

% Tipo di documento. L'uso di twoside implica che i capitoli inizino sempre con la prima pagina a sinistra, eventualmente lasciando una pagina vuota nel capitolo precedente. Se questa cosa è fastidiosa, è possibile rimuoverlo. 
\documentclass[a4paper, twoside,openright]{report}
\usepackage{siunitx}

\DeclareMathSizes{10}{10}{7}{4}

% Dimensione dei margini
\usepackage[a4paper,top=3cm,bottom=3cm,left=3cm,right=3cm]{geometry} 
% Dimensione del font
\usepackage[fontsize=13pt]{scrextend}
% Lingua del testo
\usepackage[english,italian]{babel}

\usepackage{natbib}
% \bibpunct{(}{)}{;}{a}{}{,} % to follow the A&A style

% Lingua per la bibliografia
\usepackage[fixlanguage]{babelbib}
% Codifica del testo
\usepackage[utf8]{inputenc} 
% Encoding del testo
\usepackage[T1]{fontenc}


% Permette di generare testo fittizio. Mi è stato utile 
% per capire quale sarebbe stata l'impostazione del 
% testo nella pagina prima che scrivessi un determinato paragrafo
\usepackage{lipsum}
% Per ruotare le immagini
\usepackage{rotating}
% Per modificare l'header delle pagine 
\usepackage{fancyhdr}    
\usepackage{booktabs}
\usepackage{float}

% Librerie matematiche
\usepackage{amssymb}
\usepackage{amsmath}
\usepackage{amsthm} 
\usepackage{accents}
% \usepackage{aas_macros}

% Uso delle immagini
\usepackage{graphicx}
% Uso dei colori
\usepackage[dvipsnames]{xcolor}         
% Uso dei listing per il codice
\usepackage{listings}          
% Per inserire gli hyperlinks tra i vari elementi del testo 
\usepackage{hyperref}     
% Diversi tipi di sottolineature
\usepackage[normalem]{ulem}

\usepackage{xspace}
% -----------------------------------------------------------------

% Modifica lo stile dell'header
\pagestyle{fancy}
\fancyhf{}
\lhead{\rightmark}
\rhead{\textbf{\thepage}}
\fancyfoot{}
\setlength{\headheight}{12.5pt}

% Rimuove il numero di pagina all'inizio dei capitoli
\fancypagestyle{plain}{
  \fancyfoot{}
  \fancyhead{}
  \renewcommand{\headrulewidth}{0pt}
}

% Stile del codice
\lstdefinestyle{codeStyle}{
    % Colore dei commenti
    commentstyle=\color{black},
    % Colore delle keyword
    keywordstyle=\color{black},
    % Stile dei numeri di riga
    numberstyle=\tiny\color{black},
    % Colore delle stringhe
    stringstyle=\color{black},
    % Dimensione e stile del testo
    basicstyle=\ttfamily\footnotesize,
    % newline solo ai whitespaces
    breakatwhitespace=false,     
    % newline si/no
    breaklines=true,                 
    % Posizione della caption, top/bottom 
    captionpos=b,                    
    % Mantiene gli spazi nel codice, utile per l'indentazione
    keepspaces=true,                 
    % Dove visualizzare i numeri di linea
    numbers=left,                    
    % Distanza tra i numeri di linea
    numbersep=5pt,                  
    % Mostra gli spazi bianchi o meno
    showspaces=false,                
    % Mostra gli spazi bianchi nelle stringhe
    showstringspaces=false,
    % Mostra i tab
    showtabs=false,
    % Dimensione dei tab
    tabsize=2
} \lstset{style=codeStyle}

% Stile di codice per dimensioni maggiori, in cui ho avuto bisogno di un testo più picolo (ad esempio se si vuole inserire del codice che ha linee molto lunghe). Per usare questo stile piuttosto che il precedente, usare 

% \lstset{style=longBlock}
%  % inserire il codice...
% \lstset{style=codeStyle}

% Il secondo comando consente di tornare allo stile precedente 
\lstdefinestyle{longBlock}{
    commentstyle=\color{black},
    keywordstyle=\color{black},
    numberstyle=\tiny\color{black},
    stringstyle=\color{black},
    basicstyle=\ttfamily\scriptsize,
    breakatwhitespace=false,         
    breaklines=true,                 
    captionpos=b,                    
    keepspaces=true,                 
    numbers=left,                    
    numbersep=5pt,                  
    showspaces=false,                
    showstringspaces=false,
    showtabs=false,                  
    tabsize=2
} \lstset{style=codeStyle}

% Togliendo il commento al comando che segue, si inseriscono nella bibliografia anche le fonti presenti in Bibliography.bib ma non citati direttamente con il comando \cite
% \nocite{*}

% Margini prima e dopo blocchi di codice, per avere più distanza
\lstset{aboveskip=20pt,belowskip=20pt}

% Modifica dello stile dei riferimenti, con il testo in cyano
%\hypersetup{
%    colorlinks,
%    linkcolor=CornflowerBlue,
%    citecolor=CornflowerBlue
%}

% Aggiunti definizioni, teoremi, linea e listing
\newtheorem{definition}{Definizione}[section]
\newtheorem{theorem}{Teorema}[section]
\providecommand*\definitionautorefname{Definizione}
\providecommand*\theoremautorefname{Teorema}
\providecommand*{\listingautorefname}{Listing}
\providecommand*\lstnumberautorefname{Linea}

\raggedbottom

%\newcommand{\cgs}[1]{{\textcolor{brown}[\textcolor{red}{\bf{GS: }}{ \textcolor{brown}{#1]}}}}             
%\newcommand{\cmc}[1]{{\textcolor{blue}[\textcolor{magenta}{\bf{MC: }}{ \textcolor{blue}{#1]}}}}


% User defined commands

% -----------------------------------------------------------------
\begin{document}


\tableofcontents

\chapter{Mercato dell'Energia}
\textbf{Docente: Fabrizio Pilo}\\
Dalla scelta di nazionalizzare, nel 6 dicembre del 1962, seguita ad una lunga discussione di carattere politica, il sistema dell'energia elettrica, requisendo l'imprenditoria privata, nasce l'\textbf{Ente nazionale per l'energia elettrica} (\textbf{Enel}), al quale è riservato il compito di esercitare nel territorio nazionale le attività di produzione, importazione ed esportazione, trasporto, trasformazione, distribuzione e vendita dell'energia elettrica da qualsiasi fonte prodotta.
\begin{itemize}
    \item 1950 - Comunità Europea del Carbone e dell'Acciaio (CECA);
    \item 1952 - Aderiscono vari paesi Europei;
    \item 1957 - Comunità Economica Europea (CEE);
    \item 1986 - Single European Act;
\end{itemize}
Manca un approccio di tipo federale, a livello Europeo.
Il sistema elettrico si basa du un \textit{trilemma}: sostenibilità, sicurezza di approvvigionamento, e mercato energetico Interno.
Ci sono state critiche perché, ad esempio, negli anni '90 la Germania, ma in generale l'Europa ha puntato troppo sull'aspetto economico, godendo del fatto di avere gas a basso prezzo, rendendosi dipendente da chi lo vendeva.
A un certo punto, con enormi investimenti ci si è concentrati troppo sull'aspetto sostenibilità, e altre volte è stata trascurata la sicurezza. 
Oggi si punta molto sull'aspetto sostenibilità.


\section{3/12/25 - Contesto Storico Internazionale}
La commissione Europea formula una serie di documenti, detti \textit{packages}, contenenti direttive che definiscono le regole generali entro cui si deve muovere la regolazione, e si preoccumano di garantire un mercato unitario a livello Europeo (l'ultimo è il clean-energy package). 
Il mondo energetico si divide principalmente in 
\begin{itemize}
    \item \textbf{Produzione:}\\
    In realtà ovviamente si tratta di trasformazione di energia di vario tipo in energia elettrica, non di produzione.
    \item \textbf{Trasmissione:}\\
    Inizialmente non esistevano grandi sistemi di trasmissione, il mondo elettrico nasce localmente.
    Successivamente nascono centrali termoelettriche per l'illuminazione pubblica, in DC, finché con l'avvento della AC si "sbloccano" le lunghe distanze: si può usare il trasformatore, che consente di alzare la tensione, quindi ridurre la corrente che transita nelle linee, e quindi diminuisce la perdita per effetto Joule.
    La trasmissione agisce a tensioni alte ($>30kV$ efficaci).
    \item \textbf{Distribuzione:}
    Per arrivare ai capillari serve la rete di trasmissione, che viene poi seguito, in media e bassa tensione ($<30kV$ efficaci), dalla distribuzione.
    \item \textbf{Vendita dell'energia.}
\end{itemize}
In questa classificazione, si capisce che il sistema energetico è \textbf{verticalmente integrato}.

\subsection{Monopolio e libero mercato}
Il mondo elettrico è naturalmente monopolistico, ma perché? Nasce a fine anni '80 e '90 un filone liberista che si poneva questo dubbio: perché non mettere competizione? 
L'Enel ha fatto tante cose molto importanti, però si iniziò a sentire la necessità di un cambio di paradigma. 
La competizione poteva essere una buona strategia per diminuire i prezzi, ed efficientare il sistema.\\
Oggi la produzione dell'energia è privata, fatta da soggetti privati che vogliono fare profitto, e chi ci investe ci guadagna se ha fatto bene le considerazioni su costi e opportunità. 
Tant'è che il \textit{capacity market} cerca di rendere conveniente a chi produce (o accumula) di continuare a farlo.
È grazie alla competizione che si possono rendere così efficienti e competitivi i prezzi dell'energia.

\subsubsection{Trasmissione}
Trasposta energia a $380$kV, $220$kV, $150$kV e $132$kV principalmente.
Oggi solo alcune aree sono classificabili come \textbf{monopolio naturale}, e la trasmissione è una di queste.
Ci sono essenzialmente due modi per fare comcorrenza
\begin{itemize}
    \item \textbf{Nel mercato}, dove entra di più chi mette il prezzo più basso;
    \item \textbf{Per il mercato}, dove si fa una gara pubblica, un'asta, che individui l'ente che si aggiudichi quel mercato.
\end{itemize}
I costi nella trasmissione sono talmente grandi che, in un mercato così piccolo, ogni ente avrebbe un pubblico così piccolo che non ci sarebbe convenienza: la concorrenza per mercato NON conviene economicamente nella trasmissione in Italia.
I primi tentativi furono fallimentari, quando venne istituito GNTR ad esempio, ma gestire il sistema è molto complesso, e tutto l'asset trasmissione viene quindi spostato fuori da Enel, in Terna, dove confluisce naturalmente GNTR.

\subsubsection{Distribuzione}
Si interfaccia con la trasmissione generalmente a $150$kV.
Le cabine primarie\footnote{Intorno a Cagliari si trova almeno una decina di cabine primarie.}, che trasformano questa alta tensione a media tensione, sono disposte ad anello, in modo che se non funzionasse una ci sarebbero ancora le altre (rete magliata). 
Da queste parte la media tensione con una rete radiale.
Inizialmente fu snobbata da tutti, ma c'era in realtà possibilità di guadagno, e ad oggi il mondo della distribuzione è sempre per monopolisti, che sono dei concessionari, in ambiti territoriali definiti.
Anche in questo ambito si dovrebbero fare delle gare per decidere i monopolisti, ma ci sono state discussioni per le modalità di decisione e rinnovo dei contratti senza vere e proprie gare pubbliche.
In Sardegna c'è un unico distributore, E-Distribuzione, che ha una società a se stante, indipendente, nel gruppo di Enel, che deve lavorare in modo equo per tutti. 
Tutti i soggetti come E-distribuzione, Deval, Areti etc. \textbf{DEVONO} avere nomi che li distinguano dai gruppi che hanno parte nel mercato (e che non sono quindi regolati).
Ovviamente sono soggetti regolati, che seguono e rispettano le regole stabilite dall'ente regolatore, e NON possono vendere energia.
I dati di E-Distribuzione, ad esempio, non sono acessibili a Enel.
Esistono enti attraverso i quali i vari distributori si parlano fra loro, e in Italia sono circa $120$, alcuni dei quali così piccoli che non hanno nemmeno la media tensione.

\section{Situazione odierna}
Le istituzioni Europee oggi sono 
\begin{itemize}
    \item Commissione: propone atti legislativi;
    \item Parlamento: adotta o emenda;
    \item Consiglio: accetta, rigetta o emenda;
\end{itemize}
In un sistema federale ognuno degli stati ha dell'indipendenza e del potere in più. 
Ad oggi in Eurpoa ovviamente non è così, e quindi esiste un sistema un po' complesso, per cui ogni volta che si deve fare un atto legislativo c'è un passaggio tra le tre istituzioni Europee.
Se ci sono degli emendamenti c'è un ciclo che si richiude e che ritorna al parlamento.
Se si arriva a un rigetto, c'è un comitato di conciliazione, che vede insieme le tre istituzioni per cercare un accordo. 
Acneh qui, può nuovamente esserci un accordo o un rigetto. 
Per far uscire un atto legislativo ci vuole molto lavoro e tempo.

\subsection{Come si raggiungono gli obiettivi?}
\begin{itemize}
    \item \textbf{Direttiva:} dev'essere recepita dai membri
    \item \textbf{Regolazione:} direttamente applciabile su tutto il territorio UE
    \item \textbf{Decisione:} direttamente applicabile sui soggetti
    \item \textbf{Raccomandazione:} non obbligatorio
    \item \textbf{Opinione:} organi Europei danno indicazioni non obbligatori che rendono nota l'opinione 
\end{itemize}
Il primo pacchetto Europeo, da una direttiva istituisce i Trasmission System Operator, i Distributori, le autorità di regolazione nazionali, indipendenti da governi e operatori economici, le agenzie per la cooperazione dei Regolatori (ACER), e ENTSO-E, EDSO.
Gli obiettivi li sceglie la politica, e quindi in teoria li scegliamo noi votando.
Il problema è che nel campo energetico i tempi sono lunghi e le decisioni sono costose, quindi questo sistema potrebbe essere problematico.
Ora poi le decisioni energetiche sono legate a scelte politiche anche in altri ambiti, che rende complicata anche la comunicazione. 
La stessa comunicazione di massa ha reso molto più complicate queste cose.\\
QUI ALCUNE SLIDE SU MINISTERI, AGENZIE DI REGOALZIONE, UNBUNDLING, OBIETTIVI E STRATEGIE, CODICI E LINEE GUIDA ETC.

\section{Economia}

\subsection{Microeconomia}
Da un punto di vista Governativo possono esserci mercati \textbf{liberi}, governati dai prezzi, e mercati \textbf{regolati}, come quello in cui si muove Terna, dove i prezzi dipendono da tariffe stabilite dal regolatore.
Terna gestisce a sua volta anche tutti i mercati dei servizi di dispacciamento, che in alcuni casi sono aperti. 
Gestirli ovviamente rappresenta un costo, e quanto riceve Terna nel farlo è stabilito dal regolatore.
Da un punto di vista della \textbf{trasparenza} dei prezzi, possono esistere mercati trasparenti, come il Mercato del Giorno Prima (MGP), e tutti pagano lo stesso prezzo, e mercati \textbf{opachi}, come il Mercato dei Servizi Ancillari (MSD, Monosponio), con un meccanismo un pochino più contorno, che è a scendere o a salire.
Io dico quanto costano tali servizi, ognuno fa la sua offerta, e poi come usarle ci pensa il TSO (spiegalo meglio).
Conoscere tutto come nel mercato trasparente può dare potere a qualcuno di intervenire nel mercato in maniera conveniente, quindi opaco non è necessariamente negativa come cosa.

Qui va un po' veloce, quindi seguo senza scrivere.

\subsection{Monopolio Naturale}
Il monopolio ovviamente deve essrer regolato, sennò succede quello che successe con la tratta Fiumicino-Cagliari, dove la tariffa sui non sardi era decisa da un monopolio non regolato, e c'erano prezzi folli per chi voleva andare in Sardegna in vacanza.
Nel settore elettrico, dove i monopoli naturali esistono, si fa si tutto per governare questa posizione.
In particolare, si addensano soprattutto  quando ci sono grandi \textbf{economie di scala}, quando un produttore da solo riesce a servire un mercato  a prezzi inferiori di due o più.
Reti di distribuzione gas, elettricità, linee telefoniche locali, servizi ferroviari fra città di piccola media taglia, pipeline di gas o petrolio.
Il costo medio di produzione si riduce al crescere della produzione, per monopoli naturali, assumendo costanri i cosri dei fattori di produzione.

Guarda le slides, tanto sta solo leggendo.



\section{5/12/25 - Informazione Asimmetrica: Cost of Service Regulation}
Abbiamo detto che 
\begin{equation}
    AC = \frac{F}{q}+c_v,
\end{equation}
e
\begin{equation}
    \pi = \left(\frac{F}{q}+c_v\right) q - \left(\frac{F}{q}-c_v\right) q=0
\end{equation}
Al first best il profitto era $-F$, al second best è $0$. Se includo in $F$, i costi fissi, la remunerazione del capitale, 
Ho stabilito che c'è questo prezzo: come faccio ora a equipartire a tutti gli utenti i costi che devo dare a TSO e DSO?
In una situazione ideale il regolatore ha conoscenza perfetta di tutti i costi di produzione, la tecnologia, il comportamento dell'impresa e della domanda.
Ovviamente non avviene davvero questo, l'impresa regolata ha generalmente più informazioni su tutto ciò rispetto a quante ne abbiano terze parti e regolatori. 
Questi sono dati di grandissimo valore, un asset, e sono tenuti segreti, e possono essere usati per aumentare il proprio profitto, o perseguire obiettivi manageriali.
Il rischio peggiore è quando lìimpresa riesce a catturare il regolatore, sfruttandolo a suo vantaggio.
Quest'asimmetria di informazione è combattuta dai regolatori per evitare situazioni simili.
In azienda ovviamente azioni che cerchino di sfruttare e aumentare questa asimmetria spesso sono illeciti.
Il regolatore deve stimolare l'impresa regolata:
\begin{itemize}
    \item \textbf{Efficienza:} minimizzare i costi di produzione
    \item \textbf{Attribuire i prezzi in modo efficiente}
    \item Raggiungere \textbf{l'equilibrio di bilancio}
    \item Limitare la sua capacità di esercitare \textbf{market power}
\end{itemize}
I regolatori hanno informazioni imperfette, e \textbf{moral hazard} e/o \textbf{adverse selection} sono condizioni tipiche che si manifestano quando delle parti ha un significato vantaggio informativo.
Gli approcci regolatori più tipici sono:
\begin{itemize}
    \item cost plus: regolatore permette al network di recuparare le spese più un margine autorizzato corrispondente alla remunerazione dell'investimento;
    \item Price (o revenue) cap: il regolatore stabilisce ex-ante un prezzo fisso di remunerazione del servizio;
\end{itemize}
Nel primo caso l'impresa deve dichiarare i suoi costi effettivi, e non ci sono azioni per spingere l'operatore a ottimizzare il processo.
Nel secondo al contrario, l'impresa guadagna di più se ottimizza.
Occorre capire i costi dell'impresa, fissi e variabili, e garantirne la compertura (second best).
Dopodiché va stabilita la struttura della tariffa: il regolatore non può decidere le tasse, ma può lavorare sulle tariffe.
Questa sarà applicata in base alle diverse quantità di prodotto consumato, o in modi diversi per consumatori o prodotti diversi.
Il costo totale dipende da costi operativi e a fini regolatori\footnote{Regulatory Asset Based (RAB): Riconosco quanto hai speso, e devo restituire in base a quanto hai speso.}, costi in conto capitale (tiene conto del valore del capitale usato), e altri costi.
Il guadagno annuale totale ammesso nel \textit{cost-plus} per il concessionario
\begin{equation}
    R_t = OC_t, + D_t + r \cdot RAB + TAX - F_t
\end{equation}
tiene conto di tutti i costi operativi, di manutenzione\footnote{Nota: qui può intervenire l'assimetria informativa: se ti è riconosciuto tot per la manutenzione, non è detto che tu li debba spendere. Resta però il fatto che quel tot. ti è dovuto. Sta al regolatore capire quanto è giusto riconoscere per la manutenzione.}, deprezzamenti (es. prendo un trasformatore e dopo un po' si svaluta), tasse, RAB, tasso di interesse medio, \dots
L'entita del cash flow, il denaro che torna indietro a chi ha fatto l'investimento, dev'essere ragionevole per permettere di recuperare gli investimenti a un termine non troppo lungo.
La remunerazione deve riuscire a coprire l'investimento $K_0$.
Come distribuire l'ammortamento dei costi? 
\begin{itemize}
    \item Metodo della linea retta:\\
    Investimento dingolo $K_0$, vita utile per il bene pari a $N$ anni, RAB al tempo t $RAB_t = K_0 - \Sigma_0^t D_t$
    Quindi i prezzi devono essere tali da produrre un cashflow netto ricavabile dalla \textbf{formula di Brandeis}. 
    Con questo metodo prendo tutto quel che ho speso più la remunerazione.
    Problema: i prezzi iniziali risultano essere troppo alti, e quelli finali troppo bassi (gli asset non riflettono il loro valore di mercato in un determinato momento).
    Per un'azienda con un singolo asset, quando questo viene sostituito a un costo originale che riflette i prezzi correnti, la formula di Brandeis può causare un improvviso e significativo aumento dei prezzi (\textbf{rate shock}), creando distorsioni nei consumi, e problemi politici per i regolatori.
    \item Metodo della rendita: \\
    Aiuta a spalmare meglio i profili tariffari, in modo più stabile nel tempo, un po' meno vantaggiosi per l'azienda. 
    La somma dell'ammortamento più il rendimento sul capitale residuo è mantenuta costante dall'importo appositamente scelto dell'ammortamento annuale.
    L'impresa vorrebbe avere tutto subito, tutto prima, per questo è più svantaggioso per l'impresa spalmarlo nel tempo.
\end{itemize}

Waighted Average Cost of Capital (WACC): è un indice che determina come il capitale viene remunerato, quanti soldi si fanno sul capitale. 
Si calcola usando Equity (soldi dell'azienda), o andando a chiedere prestiti (debito).
Tipicamente il tasso di ritorno sul debito è minore del tasso di ritorno su equity (che comporta più rischio).
Il rischio cresce con il rapporto debito/investimenti: chi presta i soldi vuole conseguentemente un maggior livello di remunerazione, perché se il tuo livello di indebitamento cresce, diventi un soggetto più rischioso.
Non conviene quindi indebitarsi troppo per questo motivo.
Il WACC è calcolato periodicamente dall'autorità di regolazione.

\subsection{Obiettivi}
\paragraph{Impresa} Vuole dare utili a chi ha quote dell'azienda, quindi coprire costi e remunerare creditori e investitori.
Non ha neanche una grand voglia di fare innovazione, seguire best practice, ma sono spinti a fare investimenti perché sono remunerati.
Non è detto però che ci debba essere un senso dietro l'investimento, per l'impresa basta che sia, tanto ci guadagna. 
\paragraph{Consumatori} Vogliono il minimo prezzo in relazione ai costi sostenuti, e il prezzo si basa su valutazioni economiche dell'impresa: chi me lo garantisce?
Infine ovviamente gli incentivi alla riduzione dei costi per i consumatori son sempre troppo pochi. 

\subsection{Price / Revenue CAP Regulation}
QUI CERCO DI SEGUIRE SENZA SCRIVERE PERCHé è SOPORIFERO

\subsection{TOTEX}
Finora per gli investimenti in conto capitale si è usato il sistema cost plus, mentre per la gestione del sistema si è usato il revenure cap.
TOTEX è un sistema nato in inghilterra, che tenga conto ti CAPEX e OPEX simultaneamente.
In Italia nasce un acronimo che si chiama ROSS, che sostanzialmente è un TOTEX.
Nel mondo della regolazione si hanno input regulation, dove il soggetto regolato ottiene soldi in funzione dei soldi che sostiene, attraverso il cost plus, e il revenue/price cap.
L'output regolation invece i soldi ricevuti sono indice delle prestazioni raggiunte.
Il nuovo approccio TOTEX combina costi conto capitale e costo operativi, input e output regulation, con un approccio innovativo.
Esistono sempre asimmetrie informative: il regolatore o compensa i costi sostenuti dall'impresa così come sono forniti, o ???
L'idea del TOTEX è quella di fare \textbf{profit sharing}:
\begin{itemize}
    \item Incentivi all'efficienza
    \item Riduzione del rischio 
    \item Innovazione e miglioramento continuo
    \item Trasparenza e fiducia
\end{itemize}







% Rimuovere se non si vuole la tabella delle figure
% \listoffigures

%\include{chapters/Introduzione}
\appendix

%\bibliography{chapters/Bibliografia}
\nocite{*}
\bibliographystyle{plainnat}
\bibliography{chapters/Bibliografia.bib}

\end{document}
% -----------------------------------------------------------------
